\documentclass[10pt,a4paper]{jhwhw}
\usepackage[utf8]{inputenc}
%Paquetes Necesarios
\usepackage{amsmath}
\usepackage{amsfonts}
\usepackage{amssymb}
\usepackage{makeidx}
\usepackage[spanish,es-lcroman]{babel}
\usepackage{titling}
\usepackage{amsthm}
\usepackage{enumerate}
\usepackage{tikz}
\usepackage{latexsym}
\usepackage{cite}
\usepackage{titlesec}
\usepackage{fancybox}
\usepackage{xparse}
%Quitar el identado de todos los parrafos
\setlength{\parindent}{0cm}
%Para agregar el identado en cada item de enumerate o cualquier otro, usar [\hspace{1cm}(a)]

%Comandos de Letras
\newcommand{\R}{\mathbb{R}}
\newcommand{\N}{\mathbb{N}}
\newcommand{\Z}{\mathbb{Z}}
\newcommand{\Q}{\mathbb{Q}}
\newcommand{\C}{\mathbb{C}}

%Informacion del del autor del libro y localizacion
\author{Autor: \href{https://www.facebook.com/ruller}{Raúl García}\\Pagina Web: \href{https://rull3r.github.io/}{MateTips}\\Correo: rull3r@hotmail.com}
\date{Venezuela\\ \today \\}
\title{Solucionario \\\href{https://books.google.co.ve/books?id=i4aToAEACAAJ}{Principios de Análisis Matemático - Walter Rudin}\\}
%Para el indice alfabetico
\makeindex

%Marca de agua en el documento
\usepackage{draftwatermark}
\SetWatermarkText{\textsc{\href{https://rull3r.github.io/}{Visitame en MateTips}}} % por defecto Draft 
\SetWatermarkScale{1} % para que cubra toda la página
%\SetWatermarkColor[rgb]{1,0,0} % por defecto gris claro
\SetWatermarkAngle{55} % respecto a la horizontal

\begin{document}
	
	\problema{ }\label{pro:3}
	Demostrar la Proposición 1.15 del \href{https://books.google.co.ve/books?id=i4aToAEACAAJ}{Libro}.
	Los axiomas para la multiplicación implican las proposiciones siguientes:
	
	\begin{enumerate}[(a)]
		\item Si $x \neq 0$ y $xy=xz$, entonces $y=z$
		\item Si $x \neq 0$ y $xy=x$, entonces $y=1$
		\item Si $x \neq 0$ y $xy=1$, entonces $y=\frac{1}{x}$
		\item Si $x \neq 0$ entonces $\dfrac{1}{\frac{1}{x}}=x$
	\end{enumerate}

	\solution 
	
	\begin{enumerate}[(a)]
		\item Usando la propiedad distributiva de la multiplicación $xy-xz=0$ $\Rightarrow$ $x(y-z)=0$ y como $x\neq0$ tenemos que $y=z$ \QEPD
		
		\item Usando la propiedad distributiva de la multiplicación $xy-x=0$ $\Rightarrow$ $x(y-1)=0$ y como $x\neq0$ tenemos que $y=1$ \QEPD 
		
		\item Debido a los axiomas de la multiplicación en los números reales si $x \neq 0$ entonces existe $\frac{1}{x}$ tal que $\frac{1}{x}x=1$ multiplicando por $\frac{1}{x}$ ambos miembros de la ecuación $xy=1$ $\Rightarrow$ $\frac{1}{x}xy=\frac{1}{x}$ tenemos que $y=\frac{1}{x}$ \QEPD
		
		\item Dado (a)  $\frac{1}{x}\dfrac{1}{\frac{1}{x}}=\frac{1}{x}x$ tenemos que $\dfrac{1}{\frac{1}{x}}=x$ \QEPD
		
	\end{enumerate}
	
	
\end{document}