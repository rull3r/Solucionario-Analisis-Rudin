\documentclass[10pt,a4paper]{jhwhw}
\usepackage[utf8]{inputenc}
%Paquetes Necesarios
\usepackage{amsmath}
\usepackage{amsfonts}
\usepackage{amssymb}
\usepackage{makeidx}
\usepackage[spanish,es-lcroman]{babel}
\usepackage{titling}
\usepackage{amsthm}
\usepackage{enumerate}
\usepackage{tikz}
\usepackage{latexsym}
\usepackage{cite}
\usepackage{titlesec}
\usepackage{fancybox}
\usepackage{xparse}
%Quitar el identado de todos los parrafos
\setlength{\parindent}{0cm}
%Para agregar el identado en cada item de enumerate o cualquier otro, usar [\hspace{1cm}(a)]

%Comandos de Letras
\newcommand{\R}{\mathbb{R}}
\newcommand{\N}{\mathbb{N}}
\newcommand{\Z}{\mathbb{Z}}
\newcommand{\Q}{\mathbb{Q}}
\newcommand{\C}{\mathbb{C}}

%Informacion del del autor del libro y localizacion
\author{Autor: \href{https://www.facebook.com/ruller}{Raúl García}\\Pagina Web: \href{https://rull3r.github.io/}{MateTips}\\Correo: rull3r@hotmail.com}
\date{Venezuela\\ \today \\}
\title{Solucionario \\\href{https://books.google.co.ve/books?id=i4aToAEACAAJ}{Principios de Análisis Matemático - Walter Rudin}\\}
%Para el indice alfabetico
\makeindex

%Marca de agua en el documento
\usepackage{draftwatermark}
\SetWatermarkText{\textsc{\href{https://rull3r.github.io/}{Visitame en MateTips}}} % por defecto Draft 
\SetWatermarkScale{1} % para que cubra toda la página
%\SetWatermarkColor[rgb]{1,0,0} % por defecto gris claro
\SetWatermarkAngle{55} % respecto a la horizontal

\begin{document}
	
	\problema{ }\label{pro:1}
	Si $r$ es racional ($r\neq0$) y $x$ es irracional, demuestre que $r + x$ y $rx$ son irracionales.
	\solution 
	Procederemos ambas demostraciones por reducción al absurdo. Supongamos que $r+x$ es racional, esto es $r+x=\frac{a}{b}$ para dos enteros $a$ y $b$, $r$ es racional por lo tanto $\frac{c}{d}+x=\frac{a}{b}\Rightarrow x=\frac{a}{b}-\frac{c}{d}=\frac{ad-bc}{bd}$ lo cual es un absurdo porque $x$ es un irracional, de manera análoga supondremos que $rx$ es racional, esto es $rx=\frac{a}{b}$, con $r$ siendo racional, $\frac{c}{d}x=\frac{a}{b}\Rightarrow x=\frac{ad}{bc}$ lo cual es un absurdo ya que $x$ es irracional \QEPD
	
\end{document}