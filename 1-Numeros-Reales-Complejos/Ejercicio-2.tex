\documentclass[10pt,a4paper]{jhwhw}
\usepackage[utf8]{inputenc}
%Paquetes Necesarios
\usepackage{amsmath}
\usepackage{amsfonts}
\usepackage{amssymb}
\usepackage{makeidx}
\usepackage[spanish,es-lcroman]{babel}
\usepackage{titling}
\usepackage{amsthm}
\usepackage{enumerate}
\usepackage{tikz}
\usepackage{latexsym}
\usepackage{cite}
\usepackage{titlesec}
\usepackage{fancybox}
\usepackage{xparse}
%Quitar el identado de todos los parrafos
\setlength{\parindent}{0cm}
%Para agregar el identado en cada item de enumerate o cualquier otro, usar [\hspace{1cm}(a)]

%Comandos de Letras
\newcommand{\R}{\mathbb{R}}
\newcommand{\N}{\mathbb{N}}
\newcommand{\Z}{\mathbb{Z}}
\newcommand{\Q}{\mathbb{Q}}
\newcommand{\C}{\mathbb{C}}

%Informacion del del autor del libro y localizacion
\author{Autor: \href{https://www.facebook.com/ruller}{Raúl García}\\Pagina Web: \href{https://rull3r.github.io/}{MateTips}\\Correo: rull3r@hotmail.com}
\date{Venezuela\\ \today \\}
\title{Solucionario \\\href{https://books.google.co.ve/books?id=i4aToAEACAAJ}{Principios de Análisis Matemático - Walter Rudin}\\}
%Para el indice alfabetico
\makeindex

%Marca de agua en el documento
\usepackage{draftwatermark}
\SetWatermarkText{\textsc{\href{https://rull3r.github.io/}{Visitame en MateTips}}} % por defecto Draft 
\SetWatermarkScale{1} % para que cubra toda la página
%\SetWatermarkColor[rgb]{1,0,0} % por defecto gris claro
\SetWatermarkAngle{55} % respecto a la horizontal

\begin{document}
	
	\problema{ }\label{pro:2}
	Demostrar que no existe ningún numero racional que su cuadrado sea $12$.
	\solution 
	Procederemos por reducción al absurdo. Suponemos que sí existe ese numero racional, sea $x=\sqrt{12}=\frac{a}{b}$, es decir, $12=\frac{a^2}{b^2}\Rightarrow12b^2=a^2$, $a^2$ no puede ser impar ya que $12b^2=a^2$ con $b=2s+1$ y $a=2t+1$ tenemos $12(4s^2+4s+1)=4t^2+4t+1\Rightarrow -48s^2-48s+4t^2+4t=11\Rightarrow 4(-12s^2-12s+t^2+t)=11$ lo cual es una contradicción porque 11 es primo.\\\\
	Si $a^2$ es par entonces $a=2k\Rightarrow a^2=4k^2$ luego $4k^2=12b^2\Rightarrow k^2=3b^2$ si $k^2$ es impar $b^2$ también es impar, esto es $k=2y+1$ y $b=2w+1$ entonces $4y^2+4y+1=3(4w^2+4w+1)\Rightarrow 4y^2+4y+1=12w^2+12w+3$ finalmente $4(y^2+y-3w^2-3w)=2$ lo cual es una contradicción porque 2 no es múltiplo de 4 \QEPD
	
\end{document}